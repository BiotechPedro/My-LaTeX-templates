\documentclass{beamer}
    \usetheme{Warsaw}
    \usecolortheme[RGB={42,136,230}]{structure}
    %%% {243,68,15}
    %%% {rose}
    \usefonttheme{structuresmallcapsserif}
    %%% {structurebold}
\usepackage[utf8]{inputenc}

\title{\LaTeX ~ presentations}
%%% title[abbr.]{}
\author[Sánchez]{Pedro Sánchez}
%%% \author[abbr.]{a\inst{1} \and b\inst{2}}
%%% \institute[Universities of A and B]{\inst{1}Department\\University of A \and \inst{2} \\ }
\date{\today}
%%% \date[abbr.]{}
%%% \titlegraphic[abbr.]{}
%%% \logo{\includegraphics{}}

\setbeamercovered{transparent}
%%% \setbeamercovered{invisible}

%%% %%% Automation for step-by-step visualization %%% %%%
%%% \beamerdefaultoverlayspecification{<+-| alert@+>}

\setbeamertemplate{navigation symbols}{\tiny \color{black} \insertframenumber/\inserttotalframenumber}
%%% options: {}, [default], [vertical], [only frame symbol]

\setbeamertemplate{section in toc}[ball unnumbered]

\setbeamertemplate{blocks}[rounded][shadow=true]
%%% \setbeamertemplate{blocks}[default]

\setbeamertemplate{items}[circle]
%%% options: circle, ball, square

%%% %%% Current section slides %%% %%%
%%% \AtBeginSection
%%% {
%%% \begin{frame}<beamer>
%%%     \frametitle{Outline}
%%%     \tableofcontents[currentsection]
%%% \end{frame}
%%% or
%%% \begin{frame}
%%%     \begin{structureenv}
%%%         \Large \thesection\\
%%%         rule[-.6cm]{2pt}{1.5cm}
%%%         \insertsection
%%%     \end{structureenv}
%%% \end{frame}
%%% }




\begin{document}


\begin{frame}
    \titlepage
\end{frame}


\begin{frame}
    \tableofcontents[hideallsubsections]
    %%% \tableofcontents[pausesections]
\end{frame}


\section{Introduction}
%%% \subsection{}
%%% \subsubsection{}


\begin{frame}
\frametitle{Primera transp.}
%%% \framesubtitle{}

%%% %%% Block %%% %%%
%%% \begin{block}{name}
%%% text
%%% \end{block}
%%% options: block, alertblock, exampleblock, theorem, corollary, proof(need to name the header), definition, definitions, fact, example, examples

%%% %%% Text inside a box %%% %%%
%%% \fbox{}
%%% \frame{}
%%% \setbeamercolor{name}{fg= font colour,bg= background colour} | if not set: colour = colortheme
%%% \begin{beamercolorbox}[sep=1em, wd=2cm, center, rounded=true, shadow=true]{name}
%%% text
%%% \end{beamercolorbox}

%%% %%% Columns %%% %%%
%%% \begin{columns}
%%% \column{.5\textwidth}
%%% {First column text}
%%% \column{.5\textwidth}
%%% \framebox[\textwidth]{
%%% Second column text
%%% }
%%% \end{columns}

%%% %%%  Basic overlays %%% %%%
%%% \begin{itemize}
%%% \item A \pause
%%% \item B \pause
%%% \item C
%%% \end{itemize}
%%% options: itemize, enumerate, description

%%% %%% Fancy overlays %%% %%%
%%% \begin{itemize}
%%%     \item<1-> A
%%%     \item<2-> B
%%%     \item<3-> C
%%% or
%%%     \item<1-> A
%%%     \item<3-> C
%%%     \item<2-4> B
%%% or other combinations
%%%     \item<1-> \alert<4>{A}
%%%     \item<3-> \alert<5>{C}
%%%     \item<2-4> \alert<6>{B}
%%% \end{itemize}
%%% or (robust to modifications)
%%% \begin{itemize}[<+-| alert@+>]
%%% \item Uno
%%% \item Dos
%%% \item Tres
%%% \item Cuatro
%%% and can include images
%%% \item 
%%% \begin{center}
%%%     \begin{tikzpicture}
%%%     \alt<5->
%%%     {\node[opacity=1]
%%%         {\includegraphics{}};}
%%%     {\node[opacity=.15]
%%%         {\includegraphics{}};}
%%%     \end{tikzpicture}
%%% \end{center}
%%% \end{itemize}

%%% %%% Dynamic overlays %%% %%%
%%% \begin{itemize}[<+->]
%%% \item A
%%% \item B
%%%     \begin{center}
%%%         \begin{overlayarea}{width}{height}
%%%             \only<+>{B1}
%%%             \only<+>{B2}
%%%             %\only<+>{\includegraphics{fig}}
%%%             \visible<5->{Fixed text}
%%%         \end{overlayarea}
%%%     \end{center}
%%% \begin{center}
%%%     \item<5-> \visibleP
%%% \end{center}
%%% \item C
%%% \end{itemize}

%%% %%% Also interesting %%% %%%
%%% %%% framezoom %%% %%%
%%% %%% multimedia::movie %%% %%%
%%% %%% movie15::includemovie %%% %%%


%%% %%% Alternate colors in tables %%% %%%
%%% \documentclass[xcolor=dvipsnames,table]{beamer}
%%% ...
%%% \rowcolors{2}{RoyalBlue!5}{RoyalBlue!20}
%%% \begin{tabular}{rll} \hline
%%% X1 & X2 \\ \hline
%%% 1 & A \\
%%% 2 & B \\
%%% 3 & C \\
%%% \ldots & \ldots \\ \hline
%%% \end{tabular}


text


\end{frame}


%%% %%% Leap %%% %%%
%%% \begin{frame}[label= example]{Hyperlinks}
%%%     \hyperlink<# slide>{example<# slide of destination>}{\beamerbutton{Skip to example}}
%%% options: \beamerbuttom, \beamergotobuttom, \beamerskipbuttom, \beamerreturnbuttom, {[only text]}
%%% or predefined leaps
%%% \hyperlinkpresentationstart, \hyperlinkpresentationend, \hyperlinkframestartnext, \hyperlinkframeendprev
%%% or returning to a non-printed overlay
%%% \begin{frame}<1-2>[label=name]
%%%     \begin{enumerate}<+-| alert@+>
%%%         \item A
%%%         \item B
%%%         \item C
%%%     \end{enumerate}
%%% \end{frame}
%%% \begin{frame}
%%% text
%%% \end{frame}
%%% \againframe<3>{name}


%%% %%% Text in the environment \ver+vervatim+ do not interpret any LaTex command %%% %%%
%%% \begin{frame}[fragile]
%%% \frametitle{}
%%% \begin{verbatim}
%%% %%% \begin{semiverbatim}
%%% 10 PRINT "HELLO WORLD";
%%% 20 GOTO 10
%%% \end{verbatim}
%%% \end{frame}
%%% options: fragile, plain (raw frame, no design), c (center)|b (bottom)| t(top)




\end{document}

